\section{Translating E to TEL}\label{sec:etotemp}

This section will describe a translation of the Action Langage E (which closely ressemble Event Calculus in multiples ways) into Temporal Logic.
To be more precise, into $LTL_f$.

The goal is to start from a domaine langage $E=<\Pi,\preceq,\Delta,\Phi>$ and a domain description $D=<\gamma,\eta,\tau>$, and to get a set of temporal preposition $A$ and formulas $F$, such that all formulas of $F$ are true for a finite traces $\pi$ if and only if the corresponding interpretation $H$ of $E$ is a model of $D$.

\subsection{Naive approach}

\subsubsection{Translating the Domain Language}

To make the translation, we need to start from a domain langage for wich $\Pi$ is a discrete set of timepoint, and $\preceq$ is a total ordering of $\Pi$.

The set $A$ of propositional atoms is composed of the atoms $a_i$, $f_i$, $initiation(f_i)$ and $termination(f_i)$ that will be defined later in this section.

\begin{itemize}
  \item $\Pi$ and $\preceq$ can be represented by a trace $\pi$ of the same size. (with a $\circ^{t_f}final$ or something like that)
  \item For all $A_i$ in $\Delta$ there is a propositional variable $a_i$.
  \item For all $F_i$ in $\Phi$ there is a propositional variable $f_i$.
\end{itemize}

\subsubsection{Axiomes}

\begin{itemize}
  \item Common Inertia
  $$\Box(f_i\rightarrow\circ (termination(f_i) \mathbin{\bm{\mathsf{R}}} f_i))$$
  $$\Box(\neg f_i\rightarrow\circ (initiation(f_i) \mathbin{\bm{\mathsf{R}}} \neg f_i))$$
  \item Initiation Inertia $$\Box(initiation(f_i)\rightarrow\circ (termination(f_i) \mathbin{\bm{\mathsf{R}}} f_i))$$
  \item Termination Inertia $$\Box(termination(f_i)\rightarrow\circ (initiation(f_i) \mathbin{\bm{\mathsf{R}}} \neg f_i))$$
\end{itemize}

\subsubsection{Translationg the Domain Definition}

\begin{itemize}
  \item c-proposition\footnone{c-proposition's translation also contains the definition of initiation and termination point for the temporal logic}

  $$\Box(a_i\bigwedge c_i)\rightarrow initiation(f_i)$$ or $$\Box(a_i\bigwedge c_i)\rightarrow termination(f_i)$$
  \item h-proposition

  $$\circ^{t_i}f_i$$ or $$\circ^{t_i} \neg f_i$$
  \item t-proposition

  $$\circ^{t_i}a_i$$
\end{itemize}

\iffalse

\subsection{Naive approach from iterative}

To make the translation, we need to start from a domain langage for wich $\Pi$ is a discrete set of timepoint, and $\preceq$ is a total ordering of $\Pi$. On top of that, $\Pi$ need a timepoint $T_0$ such as $\forall T_i \in \Pi, T_0\preceq T_i$.

\subsubsection{Propositions}

\begin{itemize}
  \item c-proposition

  $$\Box(A_i\bigwedge C_i)\rightarrow \circ F_i$$ or $$\Box(A_i\bigwedge C_i)\rightarrow \circ \neg F_i$$
  \item h-proposition

  $$\circ^{T_i}F_i$$ or $$\circ^{T_i} \neg F_i$$
  \item t-proposition

  $$\circ^{T_i}A_i$$
\end{itemize}

\fi
