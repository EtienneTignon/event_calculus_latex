\section{Translating E to TEL}\label{sec:etotemp}

This section will describe a translation of the Action Langage E (which closely ressemble Event Calculus in multiples ways) into Temporal Logic.
To be more precise, into $LTL_f$.

The goal is to start from a domaine langage $E=<\Pi,\preceq,\Delta,\Phi>$ and a domain description $D=<\gamma,\eta,\tau>$, and to get a set of temporal propositions $A$ and formulas $F$, such that all formulas of $F$ are true for a finite traces $\pi$ if and only if the corresponding interpretation $H$ of $E$ is a model of $D$.

\subsection{Naive approach}

\subsubsection{Translating the Domain Language}

To make the translation, we need to start from a domain langage for wich $\Pi$ is a discrete set of timepoint, and $\preceq$ is a total ordering of $\Pi$.

The set $A$ of propositional atoms is composed of the atoms $a_i$, $f_i$, $initiation(f_i)$ and $termination(f_i)$.

\begin{itemize}
  \item $\Pi$ and $\preceq$ can be represented by a trace $\pi$ of the same size. (with a $\circ^{t_f}Tail$ or something like that)
  \item For all $A_i$ in $\Delta$ there is a propositional variable $a_i$.
  \item For all $F_i$ in $\Phi$ there is a propositional variable $f_i$ and two propositional variables $initiation(f_i)$ and $termination(f_i)$.
\end{itemize}

\subsubsection{Axiomes}

\begin{itemize}
  \item Common Inertia
  $$\Box(f_i\rightarrow\circ (termination(f_i) \mathbin{\bm{\mathsf{R}}} f_i))$$
  $$\Box(\neg f_i\rightarrow\circ (initiation(f_i) \mathbin{\bm{\mathsf{R}}} \neg f_i))$$
  \item Initiation Inertia $$\Box(initiation(f_i)\rightarrow\circ f_i)$$
  \item Termination Inertia $$\Box(termination(f_i)\rightarrow\circ \neg f_i)$$
%  \item Initiation Inertia $$\Box(initiation(f_i)\rightarrow\circ (termination(f_i) \mathbin{\bm{\mathsf{R}}} f_i))$$
%  \item Termination Inertia $$\Box(termination(f_i)\rightarrow\circ (initiation(f_i) \mathbin{\bm{\mathsf{R}}} \neg f_i))$$
\end{itemize}

\subsubsection{Translationg the Domain Definition}

\begin{itemize}
  \item c-proposition\footnote{c-proposition's translation also contains the definition of initiation and termination point for the temporal logic}

  $$\Box(a_i\bigwedge c_i)\rightarrow initiation(f_i)$$ or $$\Box(a_i\bigwedge c_i)\rightarrow termination(f_i)$$
  \item h-proposition

  $$\circ^{t_i}f_i$$ or $$\circ^{t_i} \neg f_i$$
  \item t-proposition

  $$\circ^{t_i}a_i$$
\end{itemize}

\subsubsection{Example}

Let's take a domaine langage $E=<\Pi,\preceq,\Delta,\Phi>$ with :
\begin{itemize}
  \item $\Pi$ composed of 5 timepoints $t_0$ to $t_4$
  \item $\Delta$ only composed of the action $switch$
  \item $\Phi$ only composed of the action $light$
\end{itemize}

Let's take a domain description $D=<\gamma,\eta,\tau>$ with :
\begin{itemize}
  \item $\gamma$ composed of two c-proposition :
  \begin{itemize}
    \item $switch$ initiates $light$ when $\neg light$.
    \item $switch$ initiates $\neg light$ when $light$.
  \end{itemize}
  \item $\eta$ composed of two h-proposition :
  \begin{itemize}
    \item $switch$ happens-at $t_1$.
    \item $switch$ happens-at $t_3$.
  \end{itemize}
  \item $\tau$ composed of no t-proposition.
\end{itemize}

Once translated we get :
\begin{itemize}
  \item The lengh of the trace $lengh(\pi)=5$.
  \item The set of propositions $A$ composed of $switch$, $light$, $initiation(light)$ and $termination(light)$.
  \item The set of formulas $F$ composed of :
  \begin{itemize}
    \item The formulas taken from the propositions :
    \item $\Box(switch\bigwedge \neg light)\rightarrow initiation(light)$
    \item $\Box(switch\bigwedge light)\rightarrow termination(light)$
    \item $\circ^{1}switch$
    \item $\circ^{3}switch$
    \item The formulas taken from the inertia axiomes :
    \item $\Box(light\rightarrow\circ (termination(light) \mathbin{\bm{\mathsf{R}}} light))$
    \item $\Box(\neg light\rightarrow\circ (initiation(light) \mathbin{\bm{\mathsf{R}}} \neg light))$
    \item $\Box(initiation(light)\rightarrow\circ light)$
    \item $\Box(termination(light)\rightarrow\circ \neg light)$
  \end{itemize}
\end{itemize}

We can see that the original domain description had two models:
\begin{itemize}
  \item $H(light,t_0)=H(light,t_1)=H(light,t_4)=true$ and $H(light,t_2)=H(light,t_3)=false$
  \item $H(light,t_0)=H(light,t_1)=H(light,t_4)=false$ and $H(light,t_2)=H(light,t_3)=true$
\end{itemize}

Likewise, the resulting set of formulas only has two traces that respect them:
\begin{itemize}
  \item $\{light\}\{light,switch,termination(light)\}\{\}\{switch,initiation(light)\}\{light\}$
  \item $\{\}\{switch,initiation(light)\}\{light\}\{light,switch,termination(light)\}\{\}$
\end{itemize}

          \iffalse

          \subsection{Naive approach from iterative}

          To make the translation, we need to start from a domain langage for wich $\Pi$ is a discrete set of timepoint, and $\preceq$ is a total ordering of $\Pi$. On top of that, $\Pi$ need a timepoint $T_0$ such as $\forall T_i \in \Pi, T_0\preceq T_i$.

          \subsubsection{Propositions}

          \begin{itemize}
            \item c-proposition

            $$\Box(A_i\bigwedge C_i)\rightarrow \circ F_i$$ or $$\Box(A_i\bigwedge C_i)\rightarrow \circ \neg F_i$$
            \item h-proposition

            $$\circ^{T_i}F_i$$ or $$\circ^{T_i} \neg F_i$$
            \item t-proposition

            $$\circ^{T_i}A_i$$
          \end{itemize}

          \fi
