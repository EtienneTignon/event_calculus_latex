\section{Translating E to TEL}\label{sec:etotemp}

This section will describe a translation of the Action Langage E (which closely ressemble Event Calculus in multiples ways) into Temporal Logic.
To be more precise, into LTL.

\subsection{naive approach}

To make the translation, we need to start from a domain langage for wich $\Pi$ is a discrete set of timepoint, and $\preceq$ is a total ordering of $\Pi$. On top of that, $\Pi$ need a timepoint $T_0$ such as $\forall T_i \in \Pi, T_0\preceq T_i$.

\subsubsection{Axiomes}

\begin{itemize}
  \item[Common Inertia]
  $$f_i\rightarrow\circ (termination(f_i) \mathbin{\bm{\mathsf{R}}} f_i)$$
  $$\neg f_i\rightarrow\circ (initiation(f_i) \mathbin{\bm{\mathsf{R}}} \neg f_i)$$
  \item[Initiation Inertia] $$initiation(f_i)\rightarrow\circ (termination(f_i) \mathbin{\bm{\mathsf{R}}} f_i)$$
  \item[Termination Inertia] $$termination(f_i)\rightarrow\circ (initiation(f_i) \mathbin{\bm{\mathsf{R}}} \neg f_i)$$
\end{itemize}

\subsubsection{Propositions}

\begin{itemize}
  \item c-proposition

  $$\Box(A_i\bigwedge C_i)\rightarrow \circ F_i$$ or $$\box(A_i\bigwedge C_i)\rightarrow \circ \neg F_i$$
  \item h-proposition

  $$\circ^{T_i}F_i$$ or $$\circ^{T_i} \neg F_i$$
  \item t-proposition

  $$\circ^{T_i}A_i$$
\end{itemize}


\subsection{naive approach from iterative}

To make the translation, we need to start from a domain langage for wich $\Pi$ is a discrete set of timepoint, and $\preceq$ is a total ordering of $\Pi$. On top of that, $\Pi$ need a timepoint $T_0$ such as $\forall T_i \in \Pi, T_0\preceq T_i$.

\subsubsection{Propositions}

\begin{itemize}
  \item c-proposition

  $$\Box(A_i\bigwedge C_i)\rightarrow \circ F_i$$ or $$\box(A_i\bigwedge C_i)\rightarrow \circ \neg F_i$$
  \item h-proposition

  $$\circ^{T_i}F_i$$ or $$\circ^{T_i} \neg F_i$$
  \item t-proposition

  $$\circ^{T_i}A_i$$
\end{itemize}
