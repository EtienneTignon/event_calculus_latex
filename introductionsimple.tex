\section{Path Finding in event Calculus}\label{sec:introduction}

Given a graph $G = <V,E>$, a starting point $v_s$ and a goal $v_g$.

\subsection{Fluents and Events}

Event calculus is a sorted predicate calculus (with equality).
Here I will describe a very simple variant with only the most important aspect of Event Calculus.

There are the sorts:

\begin{itemize}
  \item $\mathcal{T}$ Timepoints : $\mathcal{T} = [0, 1, ... h]$.
  \item $\mathcal{F}$ Fluents : $\forall v \in V, on(v) \in \mathcal{F}$ .
  \item $\mathcal{E}$ Event : $\forall <v_o,v_d> \in E, move(v_o,v_d) \in \mathcal{E}$.
\end{itemize}

\subsection{The four predicates}

\begin{itemize}
  \item $happens \subseteq \mathcal{E}*\mathcal{T}$
  \item $holds\_at \subseteq \mathcal{F}*\mathcal{T}$
  \item $initiates \subseteq \mathcal{E}*\mathcal{F}*\mathcal{T}$
  \item $terminates \subseteq \mathcal{E}*\mathcal{F}*\mathcal{T}$
\end{itemize}

\subsection{E : Domain independant axioms}

Thoses axioms are the ones who are present no matter the domain.
All the other formulas on this paper are examples for path finding.

\subsubsection{Efect of events on fluents}

If an event $e$ happens, and this event has the effect of starting $f$, then $f$ holds the moment after the event.
\begin{equation}
  \tag{E.1}
  [happens(e,t-1)\ \land\ initiates(e,f,t-1)] \Rightarrow holds\_at(f,t)
\end{equation}

If an event $e$ happens, and this event has the effect of ending $f$, then $f$ don't holds the moment after the event.
\begin{equation}
  \tag{E.2}
  [happens(e,t-1)\ \land\ terminates(e,f,t-1)] \Rightarrow \neg holds\_at(f,t)
\end{equation}

\subsubsection{Inertia}

If a fluent $f$ holds and is not terminated, it continue to hold the next moment.
\begin{multline}
  \tag{E.3}
$$[holds\_at(f,t-1)\ \land\ \neg\exists e(happens(e,t-1)\ \land\ terminates(e,f,t-1))] \\ \Rightarrow holds\_at(f,t)$$
\end{multline}

If a fluent $f$ don't holds and is not started, it continue to not hold the next moment.
\begin{multline}
  \tag{E.4}
$$[\neg holds\_at(f,t-1)\ \land\ \neg\exists e(happens(e,t-1)\ \land\ initiates(e,f,t-1))] \\ \Rightarrow\neg holds\_at(f,t)$$
\end{multline}

\subsection{$\Sigma$ : Effects}

Those formulas defines the effects of actions. What actions make a fluent true, and what actions make a fluent false.

If an agent move, it goes to another vertex.
\begin{equation}
  \tag{$\Sigma$.1}
  \forall move(v_o,v_d) \in \mathcal{E},initiates(move(v_o,v_d),on(v_d),t)
\end{equation}

If an agent move, he left his vertex.
\begin{equation}
  \tag{$\Sigma$.2}
  \forall move(v_o,v_d) \in \mathcal{E},terminates(move(v_o,v_d),on(v_o),t)
\end{equation}

\subsection{$\Gamma$ : Observations}

Those formulas are observed facts about which fluent holds when.

The agent has his starting vertex.
\begin{equation}
  \tag{$\Gamma_i$}
  holds\_at(on(v_s),0)
\end{equation}

The agent has his goal.
\begin{equation}
  \tag{$\Gamma_f$}
  holds\_at(on(v_g),h)
\end{equation}

\subsection{$\Delta$ : Narrative}

Those formulas are observed facts about which action happens when.

In our case, we don't know about them. But we could have facts like this to force a move at some point:
\begin{equation}
  \tag{$example$}
  happens(move(v_2,v_4),2)
\end{equation}

\subsection{$\Psi$ : Actions preconditions and State constraints}

Thoses formulas are other conditions on the fluents. Things like "if those fluent holds, this one can't" for example.

For an agent to move, he must be on the vertex.
\begin{equation}
  \tag{$\Psi$.1}
  happens(move(v_o,v_d),t) \Rightarrow holds\_at(on(v_o),t)
\end{equation}

An agent is on one vertex max at each time.
\begin{equation}
  \tag{$\Psi$.2}
  [holds\_at(on(v),t)\ \land\ v\neq v']\Rightarrow \neg holds\_at(on(v'),t)
\end{equation}

\iffalse
A vertex has place for one agent only at each time.
\begin{equation}
  \tag{$\Psi$.3}
  [holds\_at(on(r,v),t)\ \land\ r\neq r']\Rightarrow \neg holds\_at(on(r',v),t)
\end{equation}
\fi

\iffalse
Agents cannot switch places
\begin{multline}
  \tag{$\Psi$.4}
  [holds\_at(on(r,v),t)\ \land\ holds\_at(on(r',v'),t)\ \land\ holds\_at(on(r',v),t+1) \\ \land\ v\neq v'\ \land\ r\neq r'] \Rightarrow \neg holds\_at(on(r,v'),t+1)
\end{multline}
\fi

\subsection{$\Omega$ : Unique names}

List of the unique names. \footnote{This is just to say that every on(something) is different to each move(something)}
\begin{equation}
  \tag{$\Omega$}
  U[move,on]
\end{equation}

\subsection{Domain description}

\begin{equation}
  \tag{$\Phi$}
  CIRC[\Sigma;initiates,terminates] \land CIRC[\Delta;happens] \land \Omega \land \Psi \land \Gamma \land E
\end{equation}
\begin{itemize}
  \item $\Sigma = \Sigma.1 \land \Sigma.2$
  \item $\Delta$ being the conjunction of all event occurrence formulas (aka the "happens facts")
  \item $\Omega$
  \item $\Psi = \Psi .1 \land \Psi .2 \land \Psi .3 \land \Psi .4$
  \item $\Gamma$ being the conjunction of all observations (aka the "holds\_at facts") counting $\Gamma_i$ and $\Gamma_f$
  \item $E = E.1 \land E.2 \land E.3 \land E.4$
\end{itemize}
