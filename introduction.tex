\section{MAPF in event Calculus}\label{sec:introduction}

Given a graph $G = <V,E>$, a list of agent $R$, a starting point and a goal for each agent $O = <A,V,V>$.

\subsection{Fluents and Events}

Event calculus is a sorted predicate calculus (with equality). There are the sorts:

\begin{itemize}
  \item Timepoints : $\mathcal{T} = [0, 1, ... h]$.
  \item Fluents : $\forall r \in R, \forall v \in V, on(r,v) \in \mathcal{F}$ .
  \item Event : $\forall r \in R, \forall <v_o,v_d> \in E, move(r,v_o,v_d) \in \mathcal{E}$.
\end{itemize}

\subsection{The four predicates}

\begin{itemize}
  \item $happens \subseteq \mathcal{A}*\mathcal{T}$
  \item $holds\_at \subseteq \mathcal{F}*\mathcal{T}$
  \item $initiates \subseteq \mathcal{A}*\mathcal{F}*\mathcal{T}$
  \item $terminates \subseteq \mathcal{A}*\mathcal{F}*\mathcal{T}$
\end{itemize}

\subsection{Domain independant axioms}

\subsubsection{Efect of events on fluents}

If an event $e$ happens, and this event has the effect of starting $f$, then $f$ holds the moment after the event.
\begin{equation}
  \tag{E.1}
  [happens(e,t-1)\ \land\ initiates(e,f,t-1)] \Rightarrow holds\_at(f,t)
\end{equation}

If an event $e$ happens, and this event has the effect of ending $f$, then $f$ don't holds the moment after the event.
\begin{equation}
  \tag{E.2}
  [happens(e,t-1)\ \land\ terminates(e,f,t-1)] \Rightarrow \neg holds\_at(f,t)
\end{equation}

\subsubsection{Inertia}

If a fluent $f$ holds and is not terminated, it continue to hold the next moment.
\begin{multline}
  \tag{E.3}
$$[holds\_at(f,t-1)\ \land\ \neg\exists e(happens(e,t-1)\ \land\ terminates(e,f,t-1))] \\ \Rightarrow holds\_at(f,t)$$
\end{multline}

If a fluent $f$ don't holds and is not started, it continue to not hold the next moment.
\begin{multline}
  \tag{E.4}
$$[\neg holds\_at(f,t-1)\ \land\ \neg\exists e(happens(e,t-1)\ \land\ initiates(e,f,t-1))] \\ \Rightarrow\neg holds\_at(f,t)$$
\end{multline}

\subsection{Domain dependant rules}

List of the unique names.
\begin{equation}
  \tag{$\Omega$}
  U[move,on]
\end{equation}

The agents have their starting vertices.
\begin{equation}
  \tag{$\Gamma$.i}
  \forall <r,v_s,v_g>\ in\ O,holds\_at(on(r,v_s),0)
\end{equation}

The agents have their goals.
\begin{equation}
  \tag{$\Gamma$.f}
  \forall <r,v_s,v_g>\ in\ O,holds\_at(on(r,v_g),h)
\end{equation}

For an agent to move, he must be on the vertice.
\begin{equation}
  \tag{$\Psi$.1}
  happens(move(r,v_o,v_d),t) \Rightarrow holds\_at(on(r,v_o),t)
\end{equation}

If an agent move, it goes to another vertice.
\begin{equation}
  \tag{$\Sigma$.1}
  \forall move(r,v_o,v_d) \in \mathcal{E},initiates(move(r,v_o,v_d),on(r,v_d),t)
\end{equation}

If an agent move, he left his vertice.
\begin{equation}
  \tag{$\Sigma$.2}
  \forall move(r,v_o,v_d) \in \mathcal{E},terminates(move(r,v_o,v_d),on(r,v_o),t)
\end{equation}

A vertice has place for one agent only at each time.
\begin{equation}
  \tag{$\Psi$.2}
  [holds\_at(on(r,v),t)\ \land\ r\neq r']\Rightarrow \neg holds\_at(on(r',v),t)
\end{equation}

An agent is on one vertice max at each time.
\begin{equation}
  \tag{$\Psi$.3}
  [holds\_at(on(r,v),t)\ \land\ v\neq v']\Rightarrow \neg holds\_at(on(r,v'),t)
\end{equation}

Agents cannot switch places
\begin{multline}
  \tag{$\Psi$.4}
  [holds\_at(on(r,v),t)\ \land\ holds\_at(on(r',v'),t)\ \land\ holds\_at(on(r',v),t+1) \\ \land\ v\neq v'\ \land\ r\neq r'] \Rightarrow \neg holds\_at(on(r,v'),t+1)
\end{multline}

\subsection{Domain description}
\begin{equation}
  \tag{$\Phi$}
  CIRC[\Sigma;initiates,terminates] \land CIRC[\Delta;happens] \land \Omega \land \Psi \land \Gamma \land E
\end{equation}
\begin{itemize}
  \item $\Sigma = \Sigma.1 \land \Sigma.2$
  \item $\Delta$ being the cojunction of all event occurence formulas (aka the "happens facts")
  \item $\Omega$
  \item $\Psi = \Psi .1 \land \Psi .2 \land \Psi .3 \land \Psi .4$
  \item $\Gamma$ being the cojunction of all observations (aka the "holds\_at facts") counting $\Gamma.i$ and $\Gamma.f$
  \item $E = E.1 \land E.2 \land E.3 \land E.4$
\end{itemize}

\subsection{Planning}

A planning problemm consist of taking $\Sigma$, $\Omega$, $\Psi$, $\Gamma$ (without $\Gamma.f$), $\Gamma.f$, and $E$ as input,
and producing as output zero or more $\Delta$ (our plan) such as $\Phi$ is consistant and $\Phi\models\Gamma.f$.
